\section{Conclusion}
\label{sec:conclusion}

This paper presents Evidence, a Python framework that gives coding
agents a principled mechanism for verifying the correctness of the
code they generate. Evidence exploits a key asymmetry: specifications
are simpler than implementations, making them easier to write
correctly and easier for humans to audit. Evidence synthesizes test
strategies from type annotations, executes a two-phase protocol that
produces minimal shrunk counterexamples, and returns structured JSON
output that agents can parse and act on without human intervention.
Integrated mutation testing quantifies specification strength, purity
analysis detects side effects, and structural inference discovers
properties automatically.

On a suite of ten modules with 21 annotated functions, Evidence
detects every seeded defect---bugs that would survive agent-generated
unit tests---and produces counterexamples small enough for both humans
and agents to diagnose at a glance.

As coding agents take on an increasing share of software development,
the gap between code generation and code verification widens.
Evidence closes this gap by shifting the verification problem from
writing correct tests to writing correct specifications---a simpler
task that aligns naturally with the division of labor between human
intent and agent execution.

Evidence is open source and available at
\url{https://github.com/emeryberger/evidence}.
